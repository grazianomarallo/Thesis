% !TEX encoding = IsoLatin

% La bibliografia, da inserirsi solo se ci sono state citazioni.
% In questo caso ricordarsi che bisogna sempre elaborare due volte il file .TEX
% perch� la prima volta viene generata la bibliografia mentre la seconda volta viene inclusa

% NOTA: citare il DOI non � obbligatorio ma MOLTO desiderabile

\begin{thebibliography}{9} % se ci sono meno di 10 citazioni
%\begin{thebibliography}{99} % se ci sono da 10 a 99 citazioni
%\begin{thebibliography}{999} % se ci sono da 100 a 999 citazioni

%%ASIACS PAPER
%citation on paper
\bibitem{asc}
%author
Vanhoef, M and Schepers, D and Piessens, F,
%title
``Discovering logical vulnerabilities in the Wi-Fi handshake using model-based testing'',
% name of the journa�
ASIA CCS 2017 - Proceedings of the 2017 ACM Asia Conference on Computer and Communications Security,
% %volume and number of the journal (may not be present)
%Vol.\ 1,
% year and month of publishing
2017,
% page of the article
pp.\ 360-371,
% DOI
\doi{110.1145/3052973}


\bibitem{krk}
% nome del progetto
OPCDE, Dubai, 7 April 2018, Vanhoef
 % URI della pagina web
\url{https://papers.mathyvanhoef.com/opcde2018-slides.pdf}



%%FUZZING PAPER
\bibitem{fzn}
Li, Jun and Zhao, Bodong and Zhang, Chao,
``Fuzzing: a survey'',
Cybersecurity,
Vol.\ 1,
2018,
pp.\ 1-13,
\doi{10.1186/s42400-018-0002-y}


\bibitem{img}
% nome del progetto
Fuzzing image
 % URI della pagina web
\url{https://medium.com/@dieswaytoofast/fuzzing-and-deep-learning-5aae84c20303}



\end{thebibliography}

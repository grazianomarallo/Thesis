% !TEX encoding = IsoLatin

% La riga soprastante serve per configurare gli editor TeXShop, TeXWorks
% e TeXstudio per gestire questo file con la codifica IsoLatin o Latin 1
% o ISO 8859-1.

% per commentare una riga mettere % al suo inizio
% per s-commentare una riga (ossia attivarla) togliere il % al suo inizio
%
\documentclass[%pdfa% formato PDF/A, obbligatorio per l'archiviazione delle tesi di Polito
cucitura%lascia margine per la rilegatura
%,twoside% per stampa fronte-retro (fortemente consigliato per tesi voluminose, opzionale per le altre)
,english
,12pt% font pi� grande (12pt) rispetto a quello normalmente usato (11pt)
,openright
]{toptesi}
%

\usepackage{hyperref}
\hypersetup{%
    pdfpagemode={UseOutlines},
    bookmarksopen,
    pdfstartview={FitH},
    colorlinks,
    linkcolor={blue},
    citecolor={red},
    urlcolor={blue}
  }
\usepackage{amsmath}
\usepackage{color}


% \documentclass[11pt,twoside,oldstyle,autoretitolo,classica,greek]{toptesi}
% \usepackage[or]{teubner}
%%%%%%%%%%%%%%%%%%%%%%%%%%%%%%%%%%%%%%%%%%%%%%%%%%%%
%
% Esempio di composizione di tesi di laurea.
%
% Questo esempio e' stato preparato inizialmente 13-marzo-1989
% e poi e' stato modificato via via che TOPtesi andava
% arricchendosi di altre possibilita'.
%
% Nel seguito laurea "quinquennale" sta anche per "specialistica" o "magistrale"

% Cambiare encoding a piacere; oppure non caricare nessun encoding se si usano
% solo caratteri a 7 bit (ASCII) nei file d'entrata.
%
\usepackage[latin1]{inputenc}% IMPORTANTE! usare codifica ISO-8859-1 per le lettere accentate

% !TEX encoding = IsoLatin

% per inserire uno spazio "fantasma" nella definizione di un'abbreviazione
\usepackage{xspace}

% per inserire un DOI senza problemi coi caratteri "strani" ivi presenti
\usepackage{doi}
\renewcommand{\doitext}{DOI }% originally was "doi:"

% per inserire correttamente le unit� di misura SI (incluse quelle binarie)
\usepackage[binary-units]{siunitx}
% se si desidera usare / invece che la potenza -1 per indicare "al secondo"
\sisetup{per-mode=symbol}

% per inserire codice di programmazione complesso
\usepackage{listings}% per inserire codice di programmazione complesso
\lstset{
basicstyle=\ttfamily,
columns=fullflexible,
xleftmargin=3ex,
breaklines,
breakatwhitespace,
escapechar=`
}

% modify some page parameters
\setlength{\parskip}{\medskipamount}

% riga orizzontale
\newcommand{\HRule}{\rule{\linewidth}{0.2mm}}
% esempio di creazione di semplici abbreviazioni
\newcommand{\ltx}{\LaTeX\xspace}
\newcommand{\txw}{TeXworks\xspace}
\newcommand{\mik}{MikTex\xspace}
\newcommand{\html}{HTML\xspace}
\newcommand{\xhtml}{XHTML\xspace}

% esempio di creazione di un'abbreviazione con un parametro (il cui uso � indicato da #1)
\newcommand{\cmd}[1]{\texttt{#1}\xspace}
% per citare un RFC, es. \rfc{822}
\newcommand{\rfc}[1]{RFC-#1\xspace}
% per citare un file (es. \file{autoexec.bat}) o una URI fittizia (es. \file{http://www.lioy.it/})
% per le URI vere usare \url o \href
\newcommand{\file}[1]{\texttt{#1}\xspace}
% per inserire codice di esempio in-line
\newcommand{\code}[1]{\lstinline|#1|}
% importante per i pathname Windows perch� non si pu� usare \ essendo un carattere riservato di Latex
\newcommand{\bs}{\textbackslash}
% definizione di un termine: formattazione ed inserimento nell'indice
\newcommand{\tdef}[1]{\textit{#1}\index{#1}}
% meta-termine, usato tipicamente nelle definizioni dei tag
\newcommand{\meta}[1]{\textit{#1}}
% abbreviazioni in inglese
\newcommand{\ie}{i.e.\xspace}
\newcommand{\eg}{e.g.\xspace}



%%% TODO: XXX My commands XXX
% Shortcuts 
\newcommand{\bt}[1]{\textbf{#1}}  %%bold text
\newcommand{\ita}[1]{\textit{#1}}  %%italic text
\newcommand{\bit}[1]{\textbf{\textit{#1}}} %%bold italic
\newcommand{\q}[1]{``#1''} %%bold italic
\newcommand{\plchld}[1]{\bt{\color{red}This chapter/section has to be write yet.}} %%placeholder
% Recurrent terms
\newcommand{\krack}{KRACK\xspace}
\newcommand{\fwh}{4-Way Handshake\xspace}

%Placeholder for images
\newcommand{\dummyfig}[1]{
  \centering
  \fbox{
    \begin{minipage}[c][0.33\textheight][c]{0.5\textwidth}
      \centering{#1}
    \end{minipage}
  }
}


\selectlanguage{english}
\begin{document}
\english

\ateneo{Politecnico di Torino}

%%% scegliere la propria facolt� (solo PRIMA dell'AA 2012-2013)
%
%\facolta[III]{Ingegneria dell'Informazione}
%\facolta[IV]{Organizzazione d'Impresa\\e Ingegneria Gestionale}
%\Materia{Remote sensing}% uso sconsigliato

%\monografia{Gestione informatizzata di un magazzino ricambi}% per la laurea triennale
\titolo{Security Analysis of the WPA2 KRACK patches}% per la laurea quinquennale e il dottorato
%\sottotitolo{Metodo dei satelliti medicei}% NON obbligatorio, per la laurea quinquennale e il dottorato
\CorsoDiLaureaIn{Master Degree in }

%%% scegliere il proprio corso
%
%\corsodilaurea{Ingegneria dell'Organizzazione d'Impresa}% per la laurea di primo e secondo livello
%\corsodilaurea{Ingegneria Logistica e della Produzione}% per la laurea di primo e secondo livello
%\corsodilaurea{Ingegneria Gestionale}% per la laurea di primo e secondo livello
\corsodilaurea{Computer Engineering}% per la laurea di primo e secondo livello
%\corsodidottorato{Meccanica}% per il dottorat

\TesiDiLaurea{Master Thesis}

\CandidateName{Candidate}
\candidato{Graziano \textsc{Marallo}}% per tutti i percorsi
%\secondocandidato{Evangelista \textsc{Torricelli}}% per la laurea magistrale solamente
%\direttore{prof. Albert Einstein}% per il dottorato
%\coordinatore{prof. Albert Einstein}% per il dottorato
\AdvisorName{Supervisors}
\relatore{prof.\ Antonio Lioy}% per la laurea e il dottorato
\secondorelatore{prof.\ Jan Tobias Muehlberg}% per la laurea magistrale
%\terzorelatore{{\tabular{@{}l}dott.\ Neil Armstrong\\prof. Maria Rossi\endtabular}}% per la laurea magistrale
%\tutore{ing.~Karl Von Braun}% per il dottorato
%\sedutadilaurea{Agosto 1615}% per la laurea quinquennale
%\esamedidottorato{Novembre 1610}% per il dottorato
\NomeAnnoAccademico{Academic Year}
%\sedutadilaurea{\textsc{December} 2018}% per la laurea triennale
\sedutadilaurea{\textsc{Academic~Year} 2018-2019}% per la laurea magistrale
%\annoaccademico{1615-1616}% solo con l'opzione classica
%\annoaccademico{2006-2007}% idem
%\ciclodidottorato{XV}% per il dottorato
\logosede{images/logopolito}



%
%\chapterbib %solo per vedere che cosa succede; e' preferibile comporre una sola bibliografia
%\AdvisorName{Supervisors}
%\newtheorem{osservazione}{Osservazione}% Standard LaTeX

%\usepackage[a-1b]{pdfx}
%\hypersetup{%
%    pdfpagemode={UseOutlines},
%    bookmarksopen,
%    pdfstartview={FitH},
%    colorlinks,
%    linkcolor={blue},
%    citecolor={green},
%    urlcolor={blue}
%  }

%
% per numerare e far comparire nell'indice anche le sezioni di quarto livello
% SCONSIGLIATO! da usarsi solo in caso di estrema necessit�
%\setcounter{secnumdepth}{4}% section-numbering-depth
%\setcounter{tocdepth}{4}% TOC-numbering-depth (TOC=Table-Of-Content)

%\setbindingcorrection{3mm}

\errorcontextlines=9

\frontespizio
\paginavuota
\newpage
%per sfruttare meglio lo spazio nella pagina
\advance\voffset -5mm
\advance\textheight 30mm

% opzionale, solo se si vuole dedicare la tesi a delle persone care
\begin{dedica}
Dedicated to my mother and father

%\textdagger\ Vi veri universum vivus vici
\end{dedica}

\sommario

Recently have been discovered that WPA2 is vulnerable to key reinstallation attacks (KRACKs).
In response, software vendors patched their implementations to prevent key reinstallations. 
However, how can we be sure those patches are correct, and indeed prevent all key reinstallations? 
What if they are flawed, and it is still possible to attack implementations? 
The goal is address these questions and perform a security analysis of patches that are supposed to prevent key reinstallation attacks.
Fuzzing technique will be applied in order to perform several analysis.

When connecting to a protected Wi-Fi network, a handshake is
executed that provides both mutual authentication and session key negotiation.
A recent discovery prove that this handshake is vulnerable to key reinstallation
attacks. In response, vendors patched their implementations to prevent key
reinstallations. However, these patches are non-trivial, and hard to get
correct. Therefore it is essential that someone audits these patches to assure
that key reinstallation attacks are indeed prevented.

More precisely, the state machine behind the handshake can be fairly complex. On
top of that, some implementations contain extra code to deal with Access Points
that do not properly follow the 802.11 standard. This further complicates an
implementation of the handshake. All combined, this makes it difficult to
reason about the correctness of a patch, meaning some patches may be flawed in
practice.

The goal of this thesis is to asses the correctness of patches. By doing that, 
different analysis will be done in order to find several bugs and possibile bug
patterns in the 4-way Handshake implementation.

\ringraziamenti

Opzionali, solo nel caso si sia ricevuto un aiuto speciale e particolarmente rilevante.

%% inserire sempre nella tesi per la laurea di I livello, perch� il nome dei tutori non � indicato sul frontespizio.
%Il lavoro descritto in questa monografia � stato svolto sotto la supervisione
%del Prof. Antonio Lioy (tutore accademico)% inserire sempre il nome del tutore accademico
% e dell'Ing. Mario Rossi (tutore aziendale)% inserire solo se la monografia � relativa ad un tirocinio.
%.

%\tablespagetrue % normalmente questa riga non serve ed e' commentata
%\figurespagetrue % normalmente questa riga non serve ed e' commentata

\indici

\mainmatter


%%% TEMPORARY CHAPTER
\chapter{History \& Work}

% !TEX encoding = IsoLatin

Questo capitolo e' stato inserito per tenere traccia delle modifiche apportate al documento durante il tempo e 
contenere i report quindicinali sul lavoro svolto.


Modifche rispetto alla versione precedente:
\begin{itemize}
  \item Fix della bibliografia. (Bibtex)
  \item Fix di alcuni typos nel documento
  \item Inseriti dei placeholder nei capitoli ancora da scrivere 
  \item Creato il capitolo introduttivo
\end{itemize}




\section{Report}\label{sec:Report}

\begin{description}
    \item[Lavoro svolto]: 
    Avendo terminato lo studio della letteratura (citata in bibliografia) che mi era stata fornita dal mio supervisor, ho concentrato la mia attenzione sulla parte pratica. Per effettuare una prima analisi del \fwh sto utilizzando un'implementazione 
    open source di un Wirless Daemon (IWD). Nello specifico i miei supervisor mi hanno suggerito di analizzare il codice sorgente contenuto nel file
    \q{iwd/unit/test-eapol.c}. In quest'ultimo essenzialmente vengono effettuati una serie di test per verificare se ogni step del \fwh viene completato con successo. Dopo un'iniziale analisi, ho individuato nel codice la chiamate alla funzione
    che testa lo scambio della PTK(Pairwise Temporary Key). Nella funzione ho individuato il punto in cui il primo pacchetto del messaggio 3 viene riempito con dati chiave dell' EAPOL frame, successivamente anche il secondo frame.
    Come punto di partenza ho pensato di sostituire ai dati che vengono passati normalmente dalla funzione originale, dati letti da un file. In questo modo facendo leggere i dati da un file e possibile sfruttare AFL e ottenere una mutazione 
    dell'input sperando di incorrere in qualche bug da poter poi analizzare.

    Per fare quanto detto � stato necessario creare un \q{harness code}. Quest'ultimo sostanzialmente � una sorta di \q{imbracatura}, o se vogliamo, un codice ausilario che permette al codice di essere sottoposto all'analisi del fuzzer.
    Le modifiche effettuate sono le seguenti:
    \begin{enumerate}
        \item modifica del main per leggere come primo argomento sulla command line un file in input
        \item impedire la lettura del frame 1 del messagio 3 dalla struttura c predefinita;
        \item fornire al fuzzer in input un file contenente dati chiave dell' EAPOL frame (usati originariamente dalla struttura);
        \item effettuare controlli sul file per evitare di incorrere in errori indesiderati (\ie file non trovato, lettura non riuscita);
        \item ripetere gli step 3 e 4 per il frame 2 del messaggio 3;
        \item rimpiazzare le chiamate alla funzioni originali con le funzioni ausilarie create;
    \end{enumerate}

    Dopo aver effettuato qualche primo test, il fuzzer non riesce ad ottenere dei risultati validi anche dopo diverse ore. Per questo motivo ho pensato di semplificare parte dei test che vengono effettuati, 
    in modo da evitare che vengano effettuati cicli inutili e modificare il codice in modo da poter ottenere risultati significativi.


    \item[Piano di lavoro]: 
    \begin{itemize}
        \item Semplificare unit test per ottenere performance migliori durante l'analisi. 
        \item Selezionare uno dei test EAPoL/WPA2 4-Way Handshake all'interno del codice e renderlo il main target dell'analisi.
        \item Fare injection di bug all'interno del codice di libreria testata (branches and paths differenti). Per esempio pensavo di inserire qualche buffer overwrite e verificare se � possibile trovarli tramite il fuzzer.
    \end{itemize}
    


\end{description}




\chapter{Introduction}

\input{chapters/intro.tex}

\chapter{Critical Analysis}

\plchld


Questo capitolo fornisce, se necessario, un'analisi critica  dei lavori precedenti sul tema trattato nella tesi

\chapter{Design}

\plchld


Discutere in questo capitolo come ? stata progettata la soluzione al problema trattato nella tesi, indicando anche se sono stati valutati vari possibili approcci o soluzioni pre-esistenti e giustificando le proprie scelte. Descrivere quindi la soluzione vera e propria.

Nel caso sia stato sviluppato del software non triviale, � buona norma dedicargli tre sezioni:
\begin{itemize}
\item architettura dell'applicazione (interazioni con gli utenti e con altri sistemi, moduli logici, flussi dati interni ed esterni);
\item manuale dello sviluppatore (descrizione dei moduli, degli algoritmi, delle interfacce e delle strutture dati);
\item manuale utente (come installare ed usare il programma, interfacce, comandi, dati in input ed in output).
\end{itemize}
Nel caso di software molto voluminoso, queste tre sezioni possono diventare tre capitoli separati.

\chapter{Results}

\plchld


Inserire in questo capitolo i risultati conseguiti, cercando di analizzarli -- se possibile -- in modo quantitativo.


\chapter{Final Conclusions}

\plchld


Qui si inseriscono brevi conclusioni sul lavoro svolto, senza ripetere inutilmente il sommario.
Si possono evidenziare i punti di forza e quelli di debolezza, i possibili sviluppi futuri o attivit? da svolgere per miglioreare i risultati.

% bibliografia scritta "a mano"
%% !TEX encoding = IsoLatin
\begin{thebibliography}{9} % less than 10 cit
%\begin{thebibliography}{99} %  10 to 99 cit


%%FUZZING PAPER
\bibitem{fzn}
Li, Jun and Zhao, Bodong and Zhang, Chao,
``Fuzzing: a survey'',
Cybersecurity,
Vol.\ 1,
2018,
pp.\ 1-13,
\doi{10.1186/s42400-018-0002-y}

%%ASIACS PAPER
%citation on paper
\bibitem{asc}
%author
Vanhoef, M and Schepers, D and Piessens, F,
%title
``Discovering logical vulnerabilities in the Wi-Fi handshake using model-based testing'',
% name of the journa�
ASIA CCS 2017 - Proceedings of the 2017 ACM Asia Conference on Computer and Communications Security,
% %volume and number of the journal (may not be present)
%Vol.\ 1,
% year and month of publishing
2017,
% page of the article
pp.\ 360-371,
% DOI
\doi{110.1145/3052973}


%%KRACK PAPER
%citation on paper
\bibitem{krk}
%author
Vanhoef, M and Piessens, F,
%title
``Key reinstallation attacks: Forcing nonce Reuse in WPA2'',
% name of the journa�
Proceedings of the ACM Conference on Computer and Communications Security,
% %volume and number of the journal (may not be present)
%Vol.\ 1,
% year and month of publishing
2017,
% page of the article
pp.\ 1313-1328,
% DOI
\doi{10.1145/3133956}




\end{thebibliography}


% se la bibliografia � stata scritta (usando Bibtex) nel file biblio.bib allora commentare la riga precedente e scommentare le due righe seguenti
\bibliographystyle{torsec}
\bibliography{biblio}



\end{document}
